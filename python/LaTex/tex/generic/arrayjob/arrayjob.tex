%%%%%%%%%%%%%%%%%%%%%%%%%%%%%% -*- Mode: Latex -*- %%%%%%%%%%%%%%%%%%%%%%%%%%%%
%% arrayjob.doc --- Documentation of the `arrayjob' package from Zhuhan Jiang
%%
%% Author          : Denis GIROU (CNRS/IDRIS - France) <Denis.Girou@idris.fr>
%% Created the     : Fri Jan  7 21:13:20 2000
%% Last mod. by    : Denis GIROU (CNRS/IDRIS - France) <Denis.Girou@idris.fr>
%% Last mod. the   : Wed May 31 20:10:39 2000
%%%%%%%%%%%%%%%%%%%%%%%%%%%%%%%%%%%%%%%%%%%%%%%%%%%%%%%%%%%%%%%%%%%%%%%%%%%%%%%

\documentclass{article}

\usepackage{arrayjob}              % Arrays management
\usepackage{fancyvrb}              % Fancy verbatims
\usepackage[width=18cm]{geometry}  % Page layout
\usepackage{graphics}              % Graphics LaTeX standard package
\usepackage{ifthen}                % Convenient conditionals in LaTeX
\usepackage{mflogo}                % For the MetaFont and MetaPost logos
\usepackage{multido}               % General loop macro
\usepackage{url}                   % URLs convenient typesetting
\usepackage{trace}

% For PSTricks examples
\newif\ifpstricks
\IfFileExists{pstricks.sty}{%
\pstrickstrue
%\usepackage{pstcol}                % PSTricks package with `color' interface
\usepackage{pst-node}              % PSTricks nodes     package
\usepackage{pst-plot}              % PSTricks plots     package
\usepackage{pst-grad}              % PSTricks gradients package
%\usepackage{pst-key}               % PSTricks interface to the `keyval' package
}{%
\usepackage{color}
}                % Color LaTeX standard package

% Definition of new colors
\definecolor{Khaki}       {rgb}{0.94,0.90,0.55}
\definecolor{LemonChiffon}{rgb}{1.,0.98,0.8}

% A unique string for TeX and LaTeX
\newcommand{\AllTeX}{%
{\rm(L\kern-.36em\raise.3ex\hbox{\sc a}\kern-.15em)%
T\kern-.1667em\lower.7ex\hbox{E}\kern-.125emX}}

% Package names
\newcommand{\AlDraTeXPackage}{\textsf{(Al)Dra\TeX}}
\newcommand{\ArrayJobPackage}{`\textsf{arrayjob}'}
\newcommand{\FormLettPackage}{`\textsf{formlett}'}
\newcommand{\IfThenPackage}{`\textsf{ifthen}'}
\newcommand{\MetaFontPackage}{\MF}
\newcommand{\MetaPostPackage}{\MP}
\newcommand{\MultiDoPackage}{`\textsf{multido}'}
\newcommand{\PSTricksPackage}{\textsf{PSTricks}}
\newcommand{\RepeatPackage}{`\textsf{repeat}'}

% Aligned labels in a description environment
\newenvironment{Description}[1]{%
\begin{list}{nothing}{\setlength{\leftmargin}{#1}
\setlength{\labelwidth}{\leftmargin}\setlength{\labelsep}{1mm}}}
{\end{list}}

% If PSTricks is not available
\newcommand{\PSTricksExampleNotShown}{%
\vspace{2mm}
\centerline{\fbox{\textbf{Example not shown here because PSTricks is not
                          available on this platform.}}}}

% Character to define verbatim strings, with escape characters /?; (from `fancyvrb')
\DefineShortVerb[commandchars=/?;]{\!}

\VerbatimFootnotes % Allow verbatim material in footnotes (from `fancyvrb')

% Some advices from Rolf Niepraschk
\newcommand{\BS}{\texttt{\symbol{`\\}}}
\newcommand{\Macro}[1]{\texttt{\BS#1}}
\newcommand{\Variable}[1]{\textnormal{\textit{\(\langle\)#1\,\(\rangle\)}}}
\newcommand{\Parameter}[1]{\texttt{\symbol{`\{}#1\symbol{`\}}}}

% To format TeX macro strings
\newcommand{\FormatTeXMacro}[1]{\Macro{#1}\hfill :}

\makeatletter

% Example environments
% (do not use in them the /?; characters, that we will be used as escape characters!)

% For highlighting some verbatim sequences (array names, macro names and comments)
\def\HLReverse#1{{%
\setlength{\fboxsep}{1pt}%
\colorbox{black}{\textcolor{white}{\textbf{#1}}}}}
\def\HLMacro#1{\BS{}def\HLMacro@i#1\@nil}
\def\HLMacro@i#1def#2\@nil{{%
\setlength{\fboxsep}{1pt}%
\colorbox{black}{\textcolor{white}{\textbf{#2}}}}}
\def\HLComment#1{\textit{#1}}

\def\Example{\FV@Environment{}{Example}}
\def\endExample{%
\end{VerbatimOut}
\Below@Example{\input{\jobname.tmp}}
\endgroup}

\def\CenterExample{\FV@Environment{}{Example}}
\def\endCenterExample{%
\end{VerbatimOut}%
\begin{center}
\Below@Example{\input{\jobname.tmp}}
\end{center}
\endgroup}

\def\SideBySideExample{\FV@Environment{}{Example}}
\def\endSideBySideExample{%
\end{VerbatimOut}%
\SideBySide@Example{\input{\jobname.tmp}}
\endgroup}

\def\FVB@Example{%
\begingroup
\FV@UseKeyValues
\parindent=0pt
\multiply\topsep by 2
\VerbatimEnvironment
\begin{VerbatimOut}[codes={\catcode`\Z=12}]{\jobname.tmp}}

\def\Below@Example#1{%
\VerbatimInput[gobble=0,commentchar=Z,commandchars=/?;,numbersep=3pt,
               frame=single,numbers=left]%
              {\jobname.tmp}
\catcode`\Z=9\relax%
% We suppress the effect of the highlighting macros
\catcode`/=0\relax%
\catcode`?=1\relax%
\catcode`;=2\relax%
\def\HLReverse##1{##1}%
\def\HLMacro##1{##1}%
\def\HLComment##1{##1}%
#1\par}

\def\SideBySide@Example#1{%
\@tempdimb=\FV@XRightMargin
\advance\@tempdimb -5mm
\begin{minipage}[c]{\@tempdimb}
  \fvset{xrightmargin=0pt}
  \catcode`\Z=9\relax%
  % We suppress the effect of the highlighting macros
  \catcode`/=0\relax%
  \catcode`?=1\relax%
  \catcode`;=2\relax%
  \def\HLReverse##1{##1}%
  \def\HLMacro##1{##1}%
  \def\HLComment##1{##1}%
  #1
\end{minipage}%
\@tempdimb=\textwidth
\advance\@tempdimb -\FV@XRightMargin
\advance\@tempdimb 5mm
\begin{minipage}[c]{\@tempdimb}
  \VerbatimInput[gobble=0,commentchar=Z,commandchars=/?;,
                 numbersep=3pt,frame=single,numbers=left,
                 xleftmargin=5mm,xrightmargin=0pt]{\jobname.tmp}
\end{minipage}}

\makeatother

% Source codes must be indented by 2 spaces
\fvset{gobble=2,frame=single,numbers=left,numbersep=1mm}

\begin{document}

\title{The \ArrayJobPackage{} package\\Management of arrays in \AllTeX}
\author{Zhuhan Jiang\\School of Computing and Mathematics\\
        University of Western Sydney\\Sydney\\Australia\\
        {\footnotesize email: zhuhan@scm.uws.edu.au}\\[5mm]
        {\footnotesize (Documentation: Denis Girou%
                       \footnote{All errors and misunderstandings are mine.}
                       \hspace{1mm}(CNRS/IDRIS -- France) --
                       Denis.Girou@idris.fr)}}
\date{Version 1.03\\May 3, 2000\\
      {\small Documentation revised May 31, 2000, edited lightly May 3, 2010 by Michael Sharpe}}

\maketitle

\begin{abstract}
  This package provides array data structures in \AllTeX, in the meaning of
the classical procedural programming languages like Fortran, Ada or C, and
macros to manipulate them. Arrays can be mono or bi-dimensional.

  This is useful for applications which require high level programming
techniques, like algorithmic graphics programmed in \TeX. This version conflicts with some macros in the {\tt amsmath} package. Version 1.04 resolves those issues. It should be used for all new work.
\end{abstract}

\tableofcontents

\section{Introduction}

  One of the big advantages of the \AllTeX{} system over common interactive
software for text processing is that it offer, also, a programming language,
which give, to people who have some knowledge in the algorithmic and
programming fields, an exceptional flexibility and power.

  Nevertheless, \TeX{} is a rather specific programming language, based on
macro expansion, which implements a lot of unusual constructs but lacks some that are very familiar in the classical procedural languages, as
\emph{arrays} to store and retrieve arbitrary pieces of data, stored in a
structured way. The main reason why they were not integrated in \TeX{} was
perhaps that arrays did not seem very useful in a language focussed on text processing (but
\MetaFontPackage{}, for instance, does has them). Nevertheless, one of the few
applications where this is straightforward to use them is to program a mailing
system, where nearly the same information is to be formatted several times,
depending of values which can be simply retrieved from an array of data. This
was the goal of the \FormLettPackage{} package \cite{FORMLETT}, written in
1993-1995 by Zhuhan Jiang for dealing with mass letters, invoices and similar
tasks of some duplicative nature. It implemented a small but powerful set of
macros to manage arrays, which have been extracted to form the present
\ArrayJobPackage{} package. The arrays can be mono or bi-dimensional%
\footnote{You can use more than two dimensions, but not in the meaning of the
classical programming languages (see page \pageref{par:DataHeight})}
and they are dynamically allocated (so we do not have to declare their
dimension statically.)%
\footnote{Stephan von Bechtolsheim \cite[Volume \textsc{III}, paragraph 20.3,
page 136]{SVB93} also demonstrated such macros for array management in the
third volume of his huge book, but it was limited to mono-dimensional
arrays.}.

  Array structures are, at the opposite of text management, often very useful
in graphic programming. This is why the \AlDraTeXPackage{} package from Eitan
Gurari \cite{ALDRATEX} (see also \cite{Gur94}) integrate such functionality
(but which is bundled inside this package and could not be extracted easily
from it), and this is also the case for the \MetaPostPackage{} package
\cite{METAPOST} (but this one do not use \TeX{} as programming language)%
\footnote{Another required feature, used in conjunction with the managements
of arrays, is a generic loop mechanism. \TeX{} offer the \Macro{loop} macro
and \LaTeX{} the \Macro{whiledo} macro of the \IfThenPackage{} package (and
also the internal \Macro{@for}, \Macro{@whilenum}, etc. macros), but a more
high-level structure is to be preferred, as the ones defined too in the
\AlDraTeXPackage{} or \MetaPostPackage{} packages. We will use in our examples
the \MultiDoPackage{} package \cite{MULTIDO}, also written by Timothy van
Zandt, but others are available, like the \RepeatPackage{} package written by
Victor Eijkhout \cite{REPEAT}.}. This fact explain why most of our examples in
this documentation will concern the area of \emph{graphic programming} (here
using the \PSTricksPackage{} package from Timothy van Zandt \cite{PSTRICKS}).

\section{Command reference}

\begin{Description}{4cm}
  \item[\FormatTeXMacro{newarray}] Define a new array.

       Syntax: !\newarray/Macro?/Variable?ArrayName;;!

       Example: !\newarray\Values!

  \item[\FormatTeXMacro{delarray}] Delete an array.

       Syntax: !\delarray/Macro?/Variable?ArrayName;;!

       Example: !\delarray\Values!

       Remarks:
       \begin{enumerate}
         \item obviously, elements of a deleted array could not be accessed
         later,
         \item take care that the elements of a deleted array are not
         themselves deleted.\newline
         So, \verb+\delarray\Data \newarray\Data \Data(I)+ can produce 
         strange behavior. Just avoid  reusing deleted arrays.
       \end{enumerate}

  \item[\FormatTeXMacro{ArrayName}] Store or get the content of the array
  element specified by the index/indices. In the last case, the content is
  inserted at the current point.

       Syntax: !/Macro?/Variable?ArrayName;;(I)=/Parameter?/Variable?Content;;! or more
               generally\newline
               \hspace*{1.3cm}!/Macro?/Variable?ArrayName;;(I1,...,In)=/Parameter?/Variable?Content;;!
               to store a value

               \hspace{1.3cm}!/Macro?/Variable?ArrayName;;(I)! or more generally
               !/Macro?/Variable?ArrayName;;(I1,...,In)! to retrieve a value

       Examples: !\Data(6)={Paul Adams}!

                 \hspace{1.75cm}!\Values(19)!

  \item[\FormatTeXMacro{readarray}] Store consecutive values in an array,
       starting from the indice one.

       Syntax: !\readarray/Parameter?/Variable?ArrayName;;=/Parameter?/Variable?Content1;&...&/Variable?ContentM;;!

       Example: !\readarray{Actors}{Louise Brooks&Marlene Dietrich&Clark Gable}!

       Remarks:
       \begin{enumerate}
         \item the values must be separated by the \textbf{\&} character,
         \item take care that the trailing spaces are significant, so the
         previous definition is different from the following one:

         !\readarray{Actors}{Louise Brooks & Marlene Dietrich & Clark Gable}!
         \item you can use \AllTeX{} macros inside the values.
       \end{enumerate}
\end{Description}

\begin{SideBySideExample}[xrightmargin=6cm]
  \newarray\Values
  \readarray{/HLReverse?Values;}{A&B&C&D}
  \verb+\Values(3)+ = `\Values(3)'
  \newarray\Actors
  \readarray{/HLReverse?Actors;}{Louise Brooks&Marlene Dietrich&Clark Gable}
  Z\par
  \verb+\Actors(2)+ = `\Actors(2)'
  \readarray{/HLReverse?Actors;}{%
  Louise Brooks & Marlene Dietrich & Clark Gable}
  Z\par
  \verb+\Actors(2)+ = `\Actors(2)'
  \Actors(4)={\textit{\underline{Ida Lupino}}}
  Z\par
  \verb+\Actors(4)+ = `\Actors(4)'
\end{SideBySideExample}

\vspace{3mm}
\noindent If you really need to trim the unnecessary left and right spaces, you must
apply a special action, like this one:

\begin{Example}
  \def\BS{\texttt{\symbol{`\\}}}

  \newarray\Strings
  \readarray{/HLReverse?Strings;}{a& b&c c  &    d dd  ddd   }

  \multido{\iString=1+1}{4}{%
    \checkStrings(\iString)%
    \BS\texttt{Strings(\iString)}=`\cachedata'\qquad}

  \makeatletter

  % /HLComment?A \TrimSpaces macro adapted from Michael J. Downes <epsmjd@ams.org>;
  % /HLComment?(posted on c.t.t. June 19, 1998);
  % /HLComment?\number`\x reads past one following space (expanding as it goes);
  \long\def\TrimSpaces#1{\expandafter\TrimSpaces@i\number`\^^00#1| |}
  % /HLComment?Remove the "0" produced by \number`\^^00, and " |" at the end.;
  \long\def\TrimSpaces@i 0#1 |{\TrimSpaces@ii\empty#1|}
  % /HLComment?" |" was removed by \TrimSpaces@i, now remove a trailing "||" or "| |";
  \long\def\TrimSpaces@ii #1|#2|{#1}

  \makeatother 

  \multido{\iString=1+1}{4}{%
    \checkStrings(\iString)%
    \BS\texttt{Strings(\iString)}=`\TrimSpaces{\cachedata}'\qquad}
\end{Example}

\begin{Description}{4cm}
  \item[\FormatTeXMacro{check}] Get the content of the array element specified
  by the indice(s) and store the result in the macro !\cachedata!

       Syntax: !\check/Variable?ArrayName;(I)! or more generally
               !\check/Variable?ArrayName;(I1,..,In)!

       Example: !\checkActors(2)!
  \item[\FormatTeXMacro{cachedata}] Macro where the content is stored after
  a !\check! request.
  \item[\FormatTeXMacro{ifemptydata}] True if the last \Macro{check} request
  has given an empty result.
\end{Description}

\newarray\Values
\readarray{Values}{A&B&C&D}

\newarray\Actors
\newarray\Dates
\newarray\Sexes
\readarray{Actors}{Louise Brooks&%
                   Marlene Dietrich&%
                   Clark Gable}
\readarray{Dates}{1906--1985&%
                  1902--1992&%
                  1901--1960}
\readarray{Sexes}{2&2&1}

\begin{SideBySideExample}[xrightmargin=6cm]
  % /HLComment?Plain TeX usage;
  \checkValues(2)%
  \verb+\Values(2)+ = `\cachedata'
  Z\par
  \checkActors(3)%
  \verb+\Actors(3)+ = `\cachedata'
  Z\par
  \checkActors(5)%
  \ifemptydata
    \verb+\Actors(5)+ not defined.
  \fi
  Z\par
  % /HLComment?LaTeX usage;
  \newcommand{\IsEmptyElement}[2]{%
  \ifthenelse{\boolean{emptydata}}{#1}{#2}}
  \checkActors(3)%
  \verb+\Actors(3)+ = \IsEmptyElement{not defined}{`\cachedata'}
  Z\par
  \checkActors(5)%
  \verb+\Actors(5)+ \IsEmptyElement{not defined}{`\cachedata'}
\end{SideBySideExample}

\begin{Description}{4cm}
  \item[\FormatTeXMacro{ifnormalindex}] See below (Default: !\normalindexfalse!).

  \item[\FormatTeXMacro{dataheight}\label{par:DataHeight}] Counter containing
  the number of elements in the first dimension, if arrays are bi-dimensional.

       Syntax: !\dataheight=/Variable?Number;!

       Remarks:
       \begin{enumerate}
         \item arrays are monodimensional when !\dataheight!~$\leq~1$,
         \item if !\normalindexfalse! (which is the default value), we have:
           $$\hspace{-3cm}\Macro{\Variable{ArrayName}}(I_1,...,I_n)
             = \Macro{\Variable{ArrayName}}(I_n + (I_{n-1}-1) *
                 \Macro{dataheight} + \cdots
                                    + (I_1-1) * \Macro{dataheight}^{n-1})$$

               and if !\normalindextrue!, we have:
           $$\hspace{-3cm}\Macro{\Variable{ArrayName}}(I_1,...,I_n)
             = \Macro{\Variable{ArrayName}}(I_1 + (I_2-1) *
                 \Macro{dataheight} + \cdots
                                    + (I_n-1) * \Macro{dataheight}^{n-1})$$
       \end{enumerate}
\end{Description}

\begin{SideBySideExample}[xrightmargin=6cm]
  \newarray\Letters
  \readarray{/HLReverse?Letters;}{A&B&C&D&E&F&G&H&I&J}
  \dataheight=5

  Z\begin{tabular}{c|c|c|c|c|c|c|c|c|}
  Z \multicolumn{1}{c}{}  & \multicolumn{1}{c}{1} & \multicolumn{1}{c}{2} &
  Z \multicolumn{1}{c}{3} & \multicolumn{1}{c}{4} & \multicolumn{1}{c}{5} \\ \cline{2-6}
  Z 1 & A & B & C & D & E \\ \cline{2-6}
  Z 2 & F & G & H & I & J \\ \cline{2-6}
  Z\end{tabular}
  Z\\[2mm]
  % /HLComment?Default is \normalindexfalse;
  \verb+\Letters(1,2)=+`\Letters(1,2)'
  Z\par
  \verb+\Letters(2,1)=+`\Letters(2,1)'
  Z\\[2mm]
  Z\begin{tabular}{c|c|c|}
  Z \multicolumn{1}{c}{} & \multicolumn{1}{c}{1} & \multicolumn{1}{c}{2} \\ \cline{2-3}
  Z 1 & A & F \\ \cline{2-3}
  Z 2 & B & G \\ \cline{2-3}
  Z 3 & C & H \\ \cline{2-3}
  Z 4 & D & I \\ \cline{2-3}
  Z 5 & E & J \\ \cline{2-3}
  Z\end{tabular}
  Z\\[2mm]
  \normalindextrue
  \verb+\Letters(1,2)=+`\Letters(1,2)'
  Z\par
  \verb+\Letters(2,1)=+`\Letters(2,1)'
\end{SideBySideExample}

\begin{Description}{4cm}
  \item[\FormatTeXMacro{ifexpandarrayelement}] Boolean macro to allow or not
  the element to be evaluated before to be stored in the array (Default:
  !\expandarrayelementfalse!).

       Syntax: !\expandarrayelementtrue! or !\expandarrayelementfalse!

       Remark: take care to the possible side effects if you store some macros
       as values of some array elements without evaluating them, as they can
       change of content later... (see the following examples).
\end{Description}

\begin{SideBySideExample}[xrightmargin=6cm]
  \newarray\Data
  \newcount\CounterP            % /HLComment?Plain TeX usage;
  \CounterP=3
  \newcounter{CounterL}         % /HLComment?LaTeX usage;
  \setcounter{CounterL}{3}
  \def\Town{Madrid}
  \Data(1)={\the\CounterP}
  \Data(2)={\the\value{CounterL}}
  \Data(3)={\Town}

  \expandarrayelementtrue
  \Data(4)={\the\CounterP}
  \Data(5)={\the\value{CounterL}}
  \Data(6)={\Town}

  \CounterP=5
  \setcounter{CounterL}{5}
  \def\Town{Roma}

  \multido{\iData=1+1}{6}{%
    \BS\texttt{Data(\iData)}=`\Data(\iData)'\\}
\end{SideBySideExample}

\vspace{3mm}
\noindent Some other macros exists for mono-dimensional arrays (and only for
them), but have little additional interest:

\begin{Description}{4cm}
  \item[\FormatTeXMacro{array}] Store or get the content of the array element
  specified by the index. In the last case, the content is inserted at the
  current point.\footnote{The macro name \verb|\array| conflicts with macros in {\tt amsmath}. This is changed in version 1.04.}

       Syntax: !\array/Parameter?/Variable?ArrayName;;(I)=/Parameter?/Variable?Content;;!
               to store a value

               \hspace{1.3cm}!\array/Parameter?/Variable?ArrayName;;(I)! to retrieve a value

       Examples: !\array{Actors}(6)={Joan Crawford}!

                 \hspace{1.75cm}!\array{Actors}(3)!

  \item[\FormatTeXMacro{clrarray}] Clear the content of the array element specified.
  A following inquiry on this element will give an empty content.

       Syntax: !\clrarray/Parameter?/Variable?ArrayName;;(I)!

       Example: !\clrarray{Actors}(2)!

  \item[\FormatTeXMacro{testarray}]  Get the content of the array element specified
  by the index and store the result in the macro !\temp@macro!%
  \footnote{In \LaTeX, this macro should be used inside a package or between
  the \verb+\makeatletter+ $\cdots$ \verb+\makeatother+ macro pair.}.

       Syntax: !\testarray/Parameter?/Variable?ArrayName;;(I)!

       Example: !\testarray{Actors}(1)\typeout{Actors(1)=`\temp@macro'}!
\end{Description}

\section{Examples}

\UndefineShortVerb{\!}          % We will use the ! character in the next examples

  We will first show some basic and easy examples, before  looking at more
advanced ones to solve some complex problems, which require more knowledge of
\TeX{} programming techniques.

\subsection{Basic examples}

  The immediate thing for which we can use arrays is to store and retrieve
information:

\vspace{2mm}
\begin{SideBySideExample}[xrightmargin=6cm]
  \newarray\Actors
  \newarray\Dates
  \newarray\Sexes

  \readarray{/HLReverse?Actors;}{Louise Brooks&Marlene Dietrich&Clark Gable}
  \readarray{/HLReverse?Dates;}{1906--1985&1902--1992&1901--1960}
  \readarray{/HLReverse?Sexes;}{2&2&1}

  \begin{description}
    \item[\Actors(1)] : \Dates(1)
    \item[\Actors(2)] : \Dates(2)
    \item[\Actors(3)] : \Dates(3)
  \end{description}
\end{SideBySideExample}

\vspace{2mm}
  But we can also use a general loop macro, as the one provided by the
\MultiDoPackage{} package, for more powerful usage and a management
independent of the number of elements:

\vspace{2mm}
\begin{SideBySideExample}[xrightmargin=6cm]
  \newcommand{\NumberActors}{3}

  \begin{description}
    \multido{\iActor=1+1}{\NumberActors}{%
      \item[\Actors(\iActor)] : \Dates(\iActor)}
  \end{description}
\end{SideBySideExample}

\vspace{2mm}
  This allow various usage in the formatting of texts. A common usage is for a
\emph{mailing} process, when we must compose some similar letters to various
people (as we said previously, this was the reason for which these macros were
developed in the \FormLettPackage{} package). Here, the usage of a programming
language allow a great flexibility, using some conditionals to format the text
differently or even to insert different pieces of text according to the person:

\newcommand{\NumberActors}{3}

\begin{CenterExample}
  \newcounter{iActor}

  \newcommand{\AccordingSexe}[2]{%
  \checkSexes(\the\value{iActor})%
  \ifthenelse{\equal{\cachedata}{1}}{#1}{#2}}

  \whiledo{\value{iActor} < \NumberActors}{%
    \stepcounter{iActor}%
    \fbox{%
      \begin{minipage}{0.985\textwidth}
        Dear \AccordingSexe{Mr}{Mrs} \Actors(\the\value{iActor}),

        I would like to tell you how I admire the great \AccordingSexe{actor}{actress}
        you are, etc.
      \end{minipage}}\\[5mm]}
\end{CenterExample}

  Nevertheless, people who know a little \TeX{} as a programming language know
that it behaviour is full of pitfalls... For instance, the following example,
which format a table according to the content of entries stored in external
arrays, can't work:

\begin{Verbatim}
  \begin{tabular}{|l|c|}
    \hline
    \multicolumn{1}{|c}{\textbf{Actors}} & \multicolumn{1}{|c|}{\textbf{Dates}} \\ \hline
    \multido{\iActor=1+1}{\NumberActors}{\Actors(\iActor) & \Dates(\iActor) \\ \hline}
  \end{tabular}
\end{Verbatim}

  This is because there is an implicit grouping of each entry in a tabular
environment and that it do not work with the grouping of the \verb+\multido+
loop. So, this must be program in a different way, storing all the content of
the table before really inserting it:

\begin{SideBySideExample}[xrightmargin=6cm,baselinestretch=0.9]
  \begin{tabular}{|l|c|}
    \hline
    \multicolumn{1}{|c}{\textbf{Actors}} &
      \multicolumn{1}{|c|}{\textbf{Dates}} \\ \hline
    \let\ListActors\empty
    \begingroup
      \let\Actors\relax
      \let\Dates\relax
      \let\\\relax
      \let\hline\relax
      \multido{\iActor=1+1}{\NumberActors}{%
        \xdef\ListActors{\ListActors
          \Actors(\iActor) & \Dates(\iActor) \\ \hline}}
    \endgroup
    \ListActors
  \end{tabular}
\end{SideBySideExample}

\subsubsection{Plot labels}

  A basic usage in graphic programming is to use arrays to retrieve the labels
to put on a drawing, as here to label the simple plot below. (Note the change from the original documentation. The axis labels used with \verb|\psaxes| must now be numeric, so it is necessary to bypass the automatic axis labeling.)
\ifpstricks                     % If PSTricks is available
\begin{CenterExample}[baselinestretch=0.9]
  \savedata{\Data}[(1,3)(2,1)(3,2)(4,1)(5,1)(6,3),(7,2)(8,1)(9,3)(10,3)(11,1)(12,1)]

  \newcommand{\PlotData}[1]{%
  \begin{pspicture}(-0.5,-0.5)(13,4)
    \psaxes[labels=none]{->}(13,4)
    \dataplot[plotstyle=dots,dotstyle=o,dotsize=0.3]{\Data}
    #1\end{pspicture}}
  \PlotData{\psaxes[showorigin=false]{->}(13,4)}

  \newarray{\Months}
  \readarray{/HLReverse?Months;}{January&February&March&April&May&June&July&August&September&%
  October&November&December}
  \newarray{\Levels}
  \readarray{/HLReverse?Levels;}{Low&Medium&High}
  %\renewcommand{\pshlabel}[1]{\tiny\Months(#1)}
  \PlotData{\psaxes[labels=y,showorigin=false]{->}(13,4)%
  \multido{\i=1+1}{12}{\rput[B](\i,-12pt){\tiny\Months(\i)}}}
  
  %\renewcommand{\psvlabel}[1]{\small\Levels(#1)}
  \PlotData{\multido{\i=1+1}{12}{\rput[B](\i,-12pt){\tiny\Months(\i)}}%
  \multido{\i=1+1}{3}{\rput[r](-12pt,\i){\small\Levels(\i)}}}
\end{CenterExample}
\else
  \PSTricksExampleNotShown
\fi                             % End \pstrickstrue

\subsubsection{Checkboard drawing}

  Of course, a bi-dimensional array directly allow to store the state of a
checkboard of one of the numerous ones existing, like the reversi game that
we will show here (in fact, the only more difficult part to understand in this
example concern the definition of the elementary piece, but as this is not
related at all to the usage of \ArrayJobPackage, this point can be forgotten
by most of the readers).

\ifpstricks                     % If PSTricks is available
\begin{SideBySideExample}[xrightmargin=6cm,baselinestretch=0.89]
  /HLMacro?\def\Black;{B}
  /HLMacro?\def\White;{W}

  /HLMacro?\def\Piece;#1#2{{%
  \psset{unit=0.8}%
  \pspicture(0,-0.1)(1,0.8)
    \pscustom[fillstyle=gradient,gradmidpoint=0,gradangle=90,
              gradbegin=#2,gradend=#1]{%
      \psbezier(0,0.2)(0.1,-0.2)(0.9,-0.2)(1,0.2)
      \psline(1,0.2)(1,0.5)
      \psbezier(1,0.5)(0.9,0.9)(0.1,0.9)(0,0.5)
      \psline(0,0.5)(0,0.2)}
      \psbezier(0,0.5)(0.1,0.1)(0.9,0.1)(1,0.5)
  \endpspicture}}

  /HLMacro?\def\PieceBlack;{\Piece{black}{gray}}
  /HLMacro?\def\PieceWhite;{\Piece{white}{gray}}

  /HLMacro?\def\CheckboardHook;{}

  \newarray\Checkboard

  /HLMacro?\def\ShowCheckboard;{%
  \dataheight=8
  \pspicture(8,8)
    \psgrid[subgriddiv=0,gridlabels=0](8,8)
    \CheckboardHook
    \multido{\iColumn=1+1,\iColumnPos=0+1}{8}{%
      \multido{\iRow=1+1,\iRowPos=7+-1}{8}{%
        \checkCheckboard(\iRow,\iColumn)%
        \ifx\cachedata\Black
          \rput(\iColumnPos.5,\iRowPos.5){\PieceBlack}
        \else
          \ifx\cachedata\White
            \rput(\iColumnPos.5,\iRowPos.5){\PieceWhite}
          \fi
        \fi}}
  \endpspicture}

  \readarray{/HLReverse?Checkboard;}{%
  W& &B& & & & &B&%
   &W&B& & & & &B&%
  W&W&B& &W& & &B&%
  W&W&B&W&W&W&B&B&%
  B& &W&W&W&B&W&B&%
  B&B&W&W&W&B&W&B&%
  B&B&B&W&W&W&W&B&%
  W&B&B&B&B&B&B&B&%
  }

  \psset{unit=0.6}
  \ShowCheckboard

  \Checkboard(2,1)={B} % /HLComment?Black move;
  /HLMacro?\def\CheckboardHook;{%
  \psframe[linestyle=none,fillstyle=hlines](0,6)(1,7)}
  % /HLComment?White pieces changed by the black move;
  \Checkboard(3,1)={B}\Checkboard(4,1)={B}
  \Checkboard(2,2)={B}\Checkboard(3,2)={B}
  \vspace{1cm}
  \ShowCheckboard
\end{SideBySideExample}
\else
  \PSTricksExampleNotShown
\fi                             % End \pstrickstrue

  Note also that, if \TeX{} is obviously not well suited to implement a
program to \emph{really} play at a game like reversi or chess, nevertheless
it can be used to solve internally some non trivial algorithmic problems,
where the usage of arrays help a lot. For such example (on the coloration of
the Truchet's tiling), see \cite{EG98}.

\subsection{Advanced examples}

  This section will show slightly more difficult to really complex examples,
and is mainly for \emph{advanced users} or for people who want to be able to
program complex tasks themselves.

\subsubsection{Example with recursion usage}

\begin{CenterExample}
  \makeatletter

  % /HLComment?The recursion macro used (from David Carlisle);
  /HLMacro?\def\Recursion;#1{%
  #1\relax
    \expandafter\@firstoftwo
  \else
    \expandafter\@secondoftwo
  \fi}

  \newcount\IndexRecursion
  \IndexRecursion=\z@

  /HLMacro?\def\PiFrac;#1{{%
  \Recursion
    {\ifnum#1>\@ne\relax}
    {\@tempcnta=#1
     \advance\@tempcnta\m@ne
     \advance\IndexRecursion\@ne
     \PiValues(\IndexRecursion)
       +\frac{\strut\displaystyle 1}{\strut\displaystyle\PiFrac{\@tempcnta}}}
    {\advance\IndexRecursion\@ne
     \PiValues(\IndexRecursion)}}}

  \makeatother

  \newarray\PiValues
  \readarray{/HLReverse?PiValues;}{3&7&15&1&292&1&1&1&2&1&3&1&14}

  \(\pi\approx\PiFrac{/HLReverse?4;}\approx\PiFrac{/HLReverse?13;}\)

  \textbf{\(\pi\) as a continued fraction}
\end{CenterExample}

\subsubsection{Structured dynamic diagrams on a grid}

  A very common case is when we must use some strutured data which are to be
drawn on a kind of abstract grid.

\ifpstricks                     % If PSTricks is available
  To draw a diagram like this one (such example illustrate the exchanges of
data between processors when using applications on multiprocessors computers
in parallel programming):

\vspace{2mm}
\begin{SideBySideExample}[xrightmargin=5.8cm]
  \psset{linestyle=none,fillstyle=solid,subgriddiv=0,gridlabels=0}
  \begin{pspicture}(-1,0)(4,4)
   % /HLComment?Line 0;
   \psframe[fillcolor=Khaki](0,3.1)(1,3.9)
   \psframe[fillcolor=LemonChiffon](1,3.1)(4,3.9)
   \rput(0,3.1){\psgrid[yunit=0.8](4,1)}
   \rput(-0.5,3.5){P0}
   \rput(0.5,3.5){A0}
   % /HLComment?Line 1;
   \psframe[fillcolor=Khaki](0,2.1)(2,2.9)
   \psframe[fillcolor=LemonChiffon](2,2.1)(4,2.9)
   \rput(0,2.1){\psgrid[yunit=0.8](4,1)}
   \rput(-0.5,2.5){P1}
   \rput(0.5,2.5){A0}\rput(1.5,2.5){A1}
   % /HLComment?Line 2;
   \psframe[fillcolor=Khaki](0,1.1)(3,1.9)
   \psframe[fillcolor=LemonChiffon](3,1.1)(4,1.9)
   \rput(0,1.1){\psgrid[yunit=0.8](4,1)}
   \rput(-0.5,1.5){P2}
   \rput(0.5,1.5){A0}\rput(1.5,1.5){A1}\rput(2.5,1.5){A2}
   % /HLComment?Line 3;
   \psframe[fillcolor=Khaki](0,0.1)(4,0.9)
   \rput(0,0.1){\psgrid[yunit=0.8](4,1)}
   \rput(-0.5,0.5){P3}
   \rput(0.5,0.5){A0}\rput(1.5,0.5){A1}
     \rput(2.5,0.5){A2}\rput(3.5,0.5){A3}
  \end{pspicture}
\end{SideBySideExample}

\vspace{2mm}
\noindent is easy but rather painful, as we must manage a lot of coordinates.
Nevertheless, even if we introduce a little more abstraction level using some
minor programming with a loop structure:
\vspace{2mm}


\begin{SideBySideExample}[xrightmargin=5.8cm]
  \psset{linestyle=none,fillstyle=solid,subgriddiv=0,gridlabels=0}
  \begin{pspicture}(-1,0)(4,4)
   \multido{\iLoop=1+1,\iRow=0+1,\iRowPos=3+-1}{4}{%
     \psframe[fillcolor=Khaki](0,\iRowPos.1)(\iLoop,\iRowPos.9)
     \psframe[fillcolor=LemonChiffon]
             (\iLoop,\iRowPos.1)(4,\iRowPos.9)
     \rput(0,\iRowPos.1){\psgrid[yunit=0.8](4,1)}
     \rput(-0.5,\iRowPos.5){P\iRow}
     \multido{\iColumn=0+1,\iLabel=1+1}{\iLoop}{%
       \rput(\iColumn.5,\iRowPos.5){A\iColumn}}}
  \end{pspicture}
\end{SideBySideExample}

\vspace{2mm}
\noindent we succeed to heavily reduce the number of lines of the code in this
specific case, but we gain nothing in genericity. If we have several such
diagrams to draw, with each that must be done differently according to the
empty and full cells of the array, we must proceed in a completely different
way, defining a \emph{generic object} which can obtain the different values of
a data structure previously filled. For such tasks, the \ArrayJobPackage{}
package allow to build very efficient and powerful tools.

\begin{Verbatim}[commandchars=/?;]
  % /HLComment?The grid;
  /HLMacro?\def\ArrayGrid;#1#2#3{{%
  % /HLComment?#1=array of data, #2=number of rows, #3=number of columns;
  \psset{dimen=middle,xunit=0.8}
  \dataheight=#3
  \pspicture(-1,-#2)(#3,0)
  \multido{\iRow=1+1,\iRowMinusOne=0+1}{#2}{%
    \rput(-1,-\iRow){\LineGrid{#1}{\iRow}{\iRowMinusOne}{#3}}}
  \endpspicture}}

  % /HLComment?One line of the grid;
  /HLMacro?\def\LineGrid;#1#2#3#4{%
  \rput(0.5,0.4){\AttributeLabel{P#3}}
  \multido{\iColumn=1+1}{#4}{%
    \expandafter\csname check#1\endcsname(#2,\iColumn)%
    \rput(\iColumn,0){%
      \ifemptydata
        \def\ColorBackground{\ColorBackgroundEmpty}%
      \else
        \ifx\cachedata\space
          \def\ColorBackground{\ColorBackgroundEmpty}%
        \else
          \def\ColorBackground{\ColorBackgroundFull}%
        \fi
      \fi
      \psframe[fillstyle=solid,fillcolor=\ColorBackground](1,0.8)
      \rput(0.5,0.4){\AttributeElement{\cachedata}}}}}
\end{Verbatim}

% The grid
\def\ArrayGrid#1#2#3{{%
% #1=array of data, #2=number of rows, #3=number of columns
\psset{dimen=middle,xunit=0.8}
\dataheight=#3
\pspicture(-1,-#2)(#3,0)
\multido{\iRow=1+1,\iRowMinusOne=0+1}{#2}{%
  \rput(-1,-\iRow){\LineGrid{#1}{\iRow}{\iRowMinusOne}{#3}}}
\endpspicture}}

% One line of the grid
\def\LineGrid#1#2#3#4{%
\rput(0.3,0.4){\AttributeLabel{P#3}}
\multido{\iColumn=1+1}{#4}{%
  \expandafter\csname check#1\endcsname(#2,\iColumn)%
  \rput(\iColumn,0){%
    \ifemptydata
      \def\ColorBackground{\ColorBackgroundEmpty}%
    \else
      \ifx\cachedata\space
        \def\ColorBackground{\ColorBackgroundEmpty}%
      \else
        \def\ColorBackground{\ColorBackgroundFull}%
      \fi
    \fi
    \psframe[fillstyle=solid,fillcolor=\ColorBackground](1,0.8)
    \rput(0.5,0.4){\AttributeElement{\cachedata}}}}}

\begin{SideBySideExample}[xrightmargin=6cm]
  % /HLComment?Default attributes;
  \def\ColorBackgroundEmpty{LemonChiffon}
  \def\ColorBackgroundFull{Khaki}
  \def\AttributeElement#1{\textcolor{red}{#1}}
  \def\AttributeLabel#1{\bf #1}

  % /HLComment?Example 1;
  \newarray\ArrayA

  % /HLComment?Data;
  \readarray{/HLReverse?ArrayA;}{%
  A0&  &  &  &%
  A0&A1&  &  &%
  A0&A1&A2&  &%
  A0&A1&A2&A3}

  \ArrayGrid{ArrayA}{4}{4}
\end{SideBySideExample}

\newarray\ArrayA

% Default attributes
\def\ColorBackgroundEmpty{LemonChiffon}
\def\ColorBackgroundFull{Khaki}
\def\AttributeElement#1{\textcolor{red}{#1}}
\def\AttributeLabel#1{\bf #1}

\begin{CenterExample}[baselinestretch=0.9]
  % /HLComment?Example 2;
  \newarray\ArrayB

  % /HLComment?Data;
  \readarray{/HLReverse?ArrayA;}{%
  A& & & &%
   & & & &%
   & & & &%
   & & & }
  \readarray{/HLReverse?ArrayB;}{%
  A& & & &%
  A& & & &%
  A& & & &%
  A& & & }

  \ArrayGrid{ArrayA}{4}{4}\hfill\ArrayGrid{ArrayB}{4}{4}
\end{CenterExample}

\newarray\ArrayB

\begin{CenterExample}[baselinestretch=0.9]
  % /HLComment?Example 3;
  \newarray\ArrayC

  % /HLComment?Data;
  \readarray{/HLReverse?ArrayA;}{%
  A0&  &  &%
  A1&  &  &%
  A2&  &  }
  \readarray{/HLReverse?ArrayB;}{%
  A0&A1&A2&%
  A0&A1&A2&%
  A0&A1&A2}
  \readarray{/HLReverse?ArrayC;}{%
  A0&  &  &%
    &A1&  &%
    &  &A2}

  % /HLComment?Attributes;
  \def\ColorBackgroundEmpty{green}
  \def\ColorBackgroundFull{red}
  \def\AttributeElement#1{\footnotesize\textcolor{white}{#1}}
  \def\AttributeLabel#1{\small\bf #1}

  \psset{unit=0.7}
  \ArrayGrid{ArrayA}{3}{3}\hfill\ArrayGrid{ArrayB}{3}{3}\hfill\ArrayGrid{ArrayC}{3}{3}
\end{CenterExample}

\begin{CenterExample}[baselinestretch=0.9]
  % /HLComment?Example 4;

  % /HLComment?Data;
  \readarray{/HLReverse?ArrayA;}{%
  A0&A1&A2&A3&%
  B0&B1&B2&B3&%
  C0&C1&C2&C3&%
  D0&D1&D2&D3}
  \readarray{/HLReverse?ArrayB;}{%
  A0&B0&C0&D0&%
  A1&B1&C1&D1&%
  A2&B2&C2&D2&%
  A3&B3&C3&D3}

  % /HLComment?Attributes;
  \def\AttributeElement#1{\textcolor{red}{#1}}
  \def\AttributeLabel#1{\Large\it #1}

  \ArrayGrid{ArrayA}{4}{4}\hfill\ArrayGrid{ArrayB}{4}{4}
\end{CenterExample}
\else
  \PSTricksExampleNotShown
\fi                             % End \pstrickstrue

\vspace{5mm}
\subsubsection{Another example of a structured dynamic diagram on a grid}

  The preceding technique is in fact relevant for a lot of problems. Here we
give another example about the drawing of linked lists (but do not fail to
note that these ones have here no internal existence).

\ifpstricks                     % If PSTricks is available
\begin{CenterExample}
  \makeatletter

  \newpsstyle{LinkedListsArrowStyle}{linecolor=red,arrowscale=2} % /HLComment?Default arrow style;

  /HLMacro?\def\LinkedListsDraw;{\@ifnextchar[\LinkedListsDraw@i{\LinkedListsDraw@i[]}}

  /HLMacro?\def\LinkedListsDraw@i;[#1]#2#3{{%
  \setkeys{psset}{#1}
  \dataheight=#3
  %
  \pst@cnta=#3
  \multiply\pst@cnta\tw@
  \advance\pst@cnta\tw@
  \pspicture(0,-#2)(\pst@cnta,0)
    \multido{\iRow=1+1}{#2}{%
      \rput(0,-\iRow){%
        \psframe(1,1)
        \psline[dimen=middle,style=LinkedListsArrowStyle]{->}(0.5,0.5)(2,0.5)}
      \multido{\iColumn=1+1}{#3}{%
        \checkLinkedListsTable(\iRow,\iColumn)%
        \ifemptydata
        \else
          \pst@cnta=\iColumn
          \advance\pst@cnta\@ne
          \pst@cntb=\@ne
          \checkLinkedListsTable(\iRow,\pst@cnta)%
          \ifemptydata                      % /HLComment?We test if it is empty;
          \else
            \ifnum\multidocount=#3          % /HLComment?We test if it is the last one;
            \else
              \pst@cntb=\z@
            \fi
          \fi
          %
          \pst@cnta=\iColumn
          \multiply\pst@cnta\tw@
          \rput(\the\pst@cnta,-\iRow)

  /HLMacro?\def\LinkedListsDraw@ii;#1#2{{%
  \rput(0,0.1){%
    \psset{unit=0.8}
    \psgrid[subgriddiv=0,gridlabels=0](2,1) % /HLComment?Element;
    \rput(0.5,0.5){#1}                      % /HLComment?Label for this element;
    \ifnum#2=\@ne
      \psline(1,0)(2,1)                     % /HLComment?Mark of the end of the list;
    \else
      \psline[style=LinkedListsArrowStyle]{->}(1.5,0.5)(2.5,0.5) % /HLComment?Link to the next element;
    \fi}}}

  \makeatother

  \newarray\LinkedListsTable
  \readarray{/HLReverse?LinkedListsTable;}{%
  11&12&13&14&%
  21&&&&%
  31&32&33&&}

  \LinkedListsDraw{3}{4}

  \vspace{1cm}
  \newpsstyle{LinkedListsArrowStyle}{linecolor=green,doubleline=true}

  \readarray{/HLReverse?LinkedListsTable;}{%
  a&&&&&%
  A&B&C&D&E&%
  0&1&2&3&4&%
  +&$\times$&&&&}

  \LinkedListsDraw{4}{5}
\end{CenterExample}
\else
  \PSTricksExampleNotShown
\fi                             % End \pstrickstrue

\subsubsection{Management of heaps and linked lists}

  Here we will demonstrate a really more complex usage, but very useful.
In fact, if the \ArrayJobPackage{} package primarily allow to manage only
arrays, it allow a lot more with some efforts. Above it, we can implement the
common internal data structures frequently used in programming, like heaps,
simple or double linked lists, trees, associative arrays, etc. Here we will
show how to define macros to build and manage heaps and simple linked lists.

\ifpstricks                     % If PSTricks is available
\vspace{1cm}                    % Poor hack to force a page break...
\begin{CenterExample}[baselinestretch=0.98]
  \makeatletter

  /HLMacro?\def\PstDebug;{\z@}              % /HLComment?For debugging, set \def\PstDebug{1};

  % /HLComment?Code for heaps management;
  % /HLComment?-------------------------;

  /HLMacro?\def\HeapMaxDepth;{100}          % /HLComment?No more than 100 elements in the heap;

  \newarray\Heap

  % /HLComment?Add an element at top of the heap;
  /HLMacro?\def\HeapPush;#1{%
  \multido{\iElem=1+1}{\HeapMaxDepth}{%
    \checkHeap(\iElem)%
    \ifemptydata
      \ifnum\PstDebug=\@ne\typeout{Push Heap(\iElem)=#1}\fi% /HLComment?Debug;
      \Heap(\iElem)={#1}%
      \multidostop
    \fi}}

  % /HLComment?Get (in the \cachedata macro) and delete the element at top of the heap;
  /HLMacro?\def\HeapPop;{%
  \Multido{\iElem=\HeapMaxDepth+-1}{\HeapMaxDepth}{%
    \checkHeap(\iElem)%
    \ifemptydata
    \else
      \ifnum\PstDebug=\@ne\typeout{Delete Heap(\iElem)=\cachedata}\fi% /HLComment?Debug;
      \Heap(\iElem)={}%
      \multidostop
    \fi}}

  % /HLComment?Print the current state of the heap;
  /HLMacro?\def\HeapPrint;{%
  \checkHeap(\@ne)%
  \ifemptydata
    \typeout{The heap is empty!}% /HLComment?Empty heap;
  \else
    \multido{\iElem=\HeapMaxDepth+-1}{\HeapMaxDepth}{%
      \checkHeap(\iElem)%
      \ifemptydata
      \else
        \typeout{Heap(\iElem)=\cachedata}%
      \fi}
  \fi
  \typeout{}}

  % /HLComment?Draw the current state of the heap;
  /HLMacro?\def\HeapDraw;{%
  % /HLComment?To compute the size of the picture;
  \Multido{\iSize=0+1}{\HeapMaxDepth}{%
    \checkHeap(\the\multidocount)%
    \ifemptydata
      \multidostop
    \fi}
  %
  \ifnum\iSize=\z@
    % /HLComment?Empty heap;
    \typeout{The heap is empty!^^J}%
  \else
    \pspicture(1,\iSize)
      \psgrid[subgriddiv=0,gridlabels=0](1,\iSize)
      \multido{\iPos=0+1}{\HeapMaxDepth}{%
        \checkHeap(\the\multidocount)%
        \ifemptydata
          \multidostop
        \else
          \rput(0.5,\iPos.5){\cachedata}
        \fi}
    \endpspicture%
  \fi}

  \makeatother

  \psset{unit=1.5}\footnotesize

  \HeapPush{Germany}\HeapPrint\HeapDraw
  \hfill
  \HeapPush{France}\HeapPrint\HeapDraw
  \hfill
  \HeapPush{Italy}\HeapPrint\HeapDraw
  \hfill
  \HeapPush{Spain}\HeapPrint\HeapDraw
  \hfill
  \HeapPop\typeout{Popped element: `\cachedata'}\HeapPrint\HeapDraw
  \hfill
  \HeapPop\typeout{Popped element: `\cachedata'}\HeapPrint\HeapDraw
  \hfill
  \HeapPop\typeout{Popped element: `\cachedata'}\HeapPrint\HeapDraw
  \hfill
  \HeapPop\typeout{Popped element: `\cachedata'}\HeapPrint\HeapDraw
\end{CenterExample}

\begin{Example}[baselinestretch=0.95]
  \makeatletter

  /HLMacro?\def\PstDebug;{\z@}%                   /HLComment?For debugging, set \def\PstDebug{1};

  % /HLComment?Code for linked lists management;
  % /HLComment?--------------------------------;

  /HLMacro?\def\LinkedListMaxDepth;{100}%         /HLComment?No more than 100 elements in the list;

  \newarray\LinkedList

  \dataheight=2%                        /HLComment?Two cells by element: the content and the pointer;

  % /HLComment?Add an element at the end of the list;
  % /HLComment?In fact, we define it here as a simple heap, but the difference is that;
  % /HLComment?we will be able to get and to delete any element in it;
  /HLMacro?\def\LinkedListAdd;#1{%
  \edef\@tempa{\@ne}% /HLComment?Pointer initialization;
  \multido{\iElem=2+1}{\LinkedListMaxDepth}{%
    \checkLinkedList(\@tempa,2)%        /HLComment?We got the content of this element;
    \ifemptydata
      % /HLComment?No pointed element, so we will insert the new one here;
      \LinkedList(\iElem,1)={#1}%
      % /HLComment?We update the pointer for the last preceding element;
      \expandarrayelementtrue%          /HLComment?We must evaluate the content of the counter;
      \LinkedList(\@tempa,2)={\iElem}%
      \expandarrayelementfalse
      \ifnum\PstDebug=\@ne
        \typeout{Add LinkedList(\@tempa)->\iElem}% /HLComment?Debug;
        \typeout{\space\space\space\space LinkedList(\iElem)=#1}% /HLComment?Debug;
      \fi
      \multidostop
    \else
      \edef\@tempa{\cachedata}%
    \fi}}

  % /HLComment?Get (in the \cachedata macro) the next element in the list;
  /HLMacro?\def\LinkedListGetNext;#1{%
  \checkLinkedList(1,2)%                /HLComment?We got the pointer for the first element;
  \edef\@tempa{\@ne}%
  \ifx\cachedata\empty
  \else
    \edef\@tempc{#1}%
    \Multido{}{\LinkedListMaxDepth}{%
      \edef\@tempb{\@tempa}%
      \edef\@tempa{\cachedata}%
      \checkLinkedList(\cachedata,1)%   /HLComment?We got the pointed element;
      \ifx\cachedata\@tempc
        % /HLComment?This is the element for which we want the next one: we got it pointer;
        % /HLComment?to the next element in the list, then the content of it;
        \checkLinkedList(\@tempa,2)%
        \ifemptydata
          \typeout{Element `#1' has no successor!^^J}%
        \else
          \checkLinkedList(\cachedata,1)%
        \fi
        \multidostop
      \else
        \checkLinkedList(\@tempa,2)%    /HLComment?We got the pointer to the next element;
        \ifemptydata
          % /HLComment?We have reach the end of the list without finding the element;
          \typeout{Element `#1' not found!^^J}%
          \multidostop%                 /HLComment?No pointed element: end of the list;
        \fi
      \fi}
  \fi}

  % /HLComment?Delete an element in the list (the code is very close to \LinkedListGetNext);
  /HLMacro?\def\LinkedListDelete;#1{%
  \checkLinkedList(1,2)%                /HLComment?We got the pointer for the first element;
  \edef\@tempa{\@ne}%
  \ifx\cachedata\empty
  \else
    \edef\@tempc{#1}%
    \multido{}{\LinkedListMaxDepth}{%
      \edef\@tempb{\@tempa}%
      \edef\@tempa{\cachedata}%
      \checkLinkedList(\cachedata,1)%   /HLComment?We got the pointed element;
      \ifx\cachedata\@tempc
        % /HLComment?This is the element to delete: we will update the pointer of;
        % /HLComment?the preceding element to the value of the pointer of the deleted one;
        \checkLinkedList(\@tempa,2)%
        \expandarrayelementtrue
        \LinkedList(\@tempb,2)={\cachedata}%
        \expandarrayelementfalse
        \ifnum\PstDebug=\@ne
          \typeout{Delete LinkedList(\@tempa)=#1}% /HLComment?Debug;
          \typeout{\space\space\space\space\space\space\space %
                          LinkedList(\@tempb)->\cachedata}% /HLComment?Debug;
        \fi
        \multidostop
      \else
        \checkLinkedList(\@tempa,2)%    /HLComment?We got the pointer to the next element;
        \ifemptydata
          % /HLComment?We have reach the end of the list without finding the element;
          \typeout{Element `#1' not found and so not deleted!^^J}%
          \multidostop%                 /HLComment?No pointed element: end of the list;
        \fi
      \fi}
  \fi}

  % /HLComment?Print the current state of the list;
  /HLMacro?\def\LinkedListPrint;{%
  \checkLinkedList(1,2)%                /HLComment?We got the pointer for the first element;
  \ifx\cachedata\empty
    % /HLComment?Empty list;
    \typeout{The list is empty!}%
  \else
    \multido{}{\LinkedListMaxDepth}{%
      \edef\@tempa{\cachedata}%
      \checkLinkedList(\cachedata,1)%   /HLComment?We got the pointed element;
      \typeout{LinkedList=\cachedata}%
      \checkLinkedList(\@tempa,2)%      /HLComment?We got the pointer to the next element;
      \ifemptydata
        \multidostop%                   /HLComment?No pointed element: end of the list;
      \fi}
  \fi
  \typeout{}}

  % /HLComment?Draw the current state of the list;
  /HLMacro?\def\LinkedListDraw;{{%
  \psset{subgriddiv=0,gridlabels=0}
  \checkLinkedList(1,2)%                /HLComment?We got the pointer for the first element;
  \ifx\cachedata\empty
    % /HLComment?Empty list;
    \typeout{The list is empty!^^J}%
  \else
    % /HLComment?To compute the size of the picture;
    \Multido{\iSize=3+3}{\LinkedListMaxDepth}{%
      \checkLinkedList(\cachedata,2)%   /HLComment?We got the pointer to the next element;
      \ifemptydata
        \multidostop
      \fi}
    %
    \checkLinkedList(1,2)%              /HLComment?We got the pointer for the first element;
    \pspicture(\iSize,1.5)
      \multido{\iPos=0+3}{\LinkedListMaxDepth}{%
        \edef\@tempa{\cachedata}%
        \checkLinkedList(\cachedata,1)% /HLComment?We got the pointed element;
        \rput(\iPos,0){%
          \psgrid(2,1)
          \rput(0.5,0.5){\cachedata}
          \ifnum\iPos=\z@
          \else
            \rput(-1.5,0.5){\psline[linecolor=red,arrowscale=2]{->}(1.5,0)}
          \fi}
        \checkLinkedList(\@tempa,2)%    /HLComment?We got the pointer to the next element;
        \ifemptydata
          \rput(\iPos,0){\psline[linecolor=red](1,0)(2,1)} % /HLComment?End of the list;
          \multidostop%                 /HLComment?No pointed element: end of the list;
        \fi}
    \endpspicture%
  \fi}}

  \makeatother

  \psset{unit=1.5}

  \LinkedListPrint\LinkedListDraw

  \LinkedListAdd{Germany}\LinkedListPrint\LinkedListDraw

  \LinkedListAdd{France}\LinkedListPrint\LinkedListDraw

  \LinkedListAdd{Italy}\LinkedListPrint\LinkedListDraw

  \LinkedListAdd{Spain}\LinkedListPrint\LinkedListDraw

  \LinkedListGetNext{France}\typeout{Next element after `France' is: `\cachedata'}%
  \LinkedListGetNext{Germany}\typeout{Next element after `Germany is: `\cachedata'}%
  \LinkedListGetNext{Spain}
  \LinkedListGetNext{Unknown}

  \LinkedListDelete{Unknown}

  \LinkedListDelete{France}\LinkedListPrint\LinkedListDraw

  \LinkedListDelete{Germany}\LinkedListPrint\LinkedListDraw

  \LinkedListDelete{Spain}\LinkedListPrint\LinkedListDraw

  \LinkedListDelete{Italy}\LinkedListPrint\LinkedListDraw
\end{Example}

\makeatletter

\def\PstDebug{\z@}%                 For debugging, set \def\PstDebug{1}

\def\LinkedListMaxDepth{100}%       No more than 100 elements in the heap

\newarray\LinkedList

\dataheight=2%                      Two cells by element: the content and the pointer

% Delete an element in the list
\def\LinkedListDelete#1{%
\checkLinkedList(1,2)%              We got the pointer for the first element
\edef\@tempa{\@ne}%
\ifx\cachedata\empty
\else
  \edef\@tempc{#1}%
  \multido{}{\LinkedListMaxDepth}{%
    \edef\@tempb{\@tempa}%
    \edef\@tempa{\cachedata}%
    \checkLinkedList(\cachedata,1)% We got the pointed element
    \ifx\cachedata\@tempc
      % This is the element to delete: we will update the pointer of
      % the preceding element to the value of the pointer of the deleted one
      \checkLinkedList(\@tempa,2)%
      \expandarrayelementtrue
      \LinkedList(\@tempb,2)={\cachedata}%
      \expandarrayelementfalse
      \ifnum\PstDebug=\@ne
        \typeout{Delete LinkedList(\@tempa)=#1}%Debug
        \typeout{\space\space\space\space\space\space\space %
                                   LinkedList(\@tempb)->\cachedata}%Debug
      \fi
      \multidostop
    \else
      \checkLinkedList(\@tempa,2)%  We got the pointer to the next element
      \ifemptydata
        % We have reach the end of the list without finding the element
        \typeout{Element `#1' not found and so not deleted!^^J}%
        \multidostop%               No pointed element: end of the list
      \fi
    \fi}
\fi}

% Print the current state of the list
\def\LinkedListPrint{%
\checkLinkedList(1,2)%              We got the pointer for the first element
\ifx\cachedata\empty
  % Empty list
  \typeout{The list is empty!}%
\else
  \multido{}{\LinkedListMaxDepth}{%
    \edef\@tempa{\cachedata}%
    \checkLinkedList(\cachedata,1)% We got the pointed element
    \typeout{LinkedList=\cachedata}%
    \checkLinkedList(\@tempa,2)%    We got the pointer to the next element
    \ifemptydata
      \multidostop%                 No pointed element: end of the list
    \fi}
\fi
\typeout{}}

% Draw the current state of the list
\def\LinkedListDraw{{%
\psset{subgriddiv=0,gridlabels=0}
\checkLinkedList(1,2)%                We got the pointer for the first element
\ifx\cachedata\empty
  % Empty list
  \typeout{The list is empty!^^J}%
\else
  % To compute the size of the picture
  \Multido{\iSize=3+3}{\LinkedListMaxDepth}{%
    \checkLinkedList(\cachedata,2)%   We got the pointer to the next element
    \ifemptydata
      \multidostop
    \fi}
  %
  \checkLinkedList(1,2)%              We got the pointer for the first element
  \pspicture(\iSize,1.5)
    \multido{\iPos=0+3}{\LinkedListMaxDepth}{%
      \edef\@tempa{\cachedata}%
      \checkLinkedList(\cachedata,1)% We got the pointed element
      \rput(\iPos,0){%
        \psgrid(2,1)
        \rput(0.5,0.5){\cachedata}
        \ifnum\iPos=\z@
        \else
          \rput(-1.5,0.5){\psline[linecolor=red,arrowscale=2]{->}(1.5,0)}
        \fi}
      \checkLinkedList(\@tempa,2)%    We got the pointer to the next element
      \ifemptydata
        \rput(\iPos,0){\psline[linecolor=red](1,0)(2,1)} % End of the list
        \multidostop%                 No pointed element: end of the list
      \fi}
  \endpspicture%
\fi}}

\makeatother

% We clear the linked list
\multido{\iElem=1+1}{\LinkedListMaxDepth}{%
  \checkLinkedList(\iElem,1)%
  \ifemptydata
  \else
    \LinkedList(\iElem,1)={}%
    \LinkedList(\iElem,2)={}%
  \fi}

\vspace{2mm}
  We can write an extended version of the inclusion of an element, where we do
not still store each element at the end of the list, but insert it at a
special position. For instance, we can decide to use the list to sort the
elements, inserting a new one at it sorted position. In the next example, we
will manage integer numbers in this way (of course, we can do the same thing
with arbitrary strings, but it would be more difficult to program in \TeX{},
mainly if we want to be able to use accentuated letters).

\vspace{2mm}
\begin{Example}
  \makeatletter

  % /HLComment?Add an element at the end of the list, linked in it ordered place;
  % /HLComment?(this version work only with contents as numbers);
  /HLMacro?\def\LinkedListSortedAdd;#1{%
  % /HLComment?Initializations;
  \edef\@tempa{\@ne}%
  \edef\@tempb{\@ne}%
  \edef\@tempc{\z@}%
  %
  \multido{\iElem=2+1}{\LinkedListMaxDepth}{%
    \checkLinkedList(\@tempa,2)% /HLComment?We got the pointer of this element;
    \ifemptydata
      % /HLComment?No pointed element, so we will insert the new one here;
      \LinkedList(\iElem,1)={#1}%
      % /HLComment?We update the pointer for the last element;
      \expandarrayelementtrue
      \LinkedList(\@tempb,2)={\iElem}%
      \ifnum\PstDebug=\@ne
        \typeout{Add LinkedList(\@tempb)->\iElem}% /HLComment?Debug;
        \typeout{\space\space\space\space LinkedList(\iElem)=#1}% /HLComment?Debug;
      \fi
      % /HLComment?We update the pointer of the new element to the position;
      % /HLComment?of the next greater one, if it exist;
      \ifnum\@tempc=\z@
      \else
        \LinkedList(\iElem,2)={\@tempc}%
        \ifnum\PstDebug=\@ne
          \typeout{Update LinkedList(\iElem)->\@tempc}% /HLComment?Debug;
        \fi
      \fi
      \expandarrayelementfalse
      \multidostop
    \else
      \edef\@tempa{\cachedata}%
      \checkLinkedList(\cachedata,1)%
      \ifnum#1<\cachedata
        \edef\@tempc{\@tempa}% /HLComment?New element less than this one;
      \else
        \edef\@tempb{\@tempa}% /HLComment?New element greater or equal than this one;
      \fi
    \fi}}

  \makeatother

  \LinkedListPrint

  \LinkedListSortedAdd{732}\LinkedListPrint\LinkedListDraw

  \LinkedListSortedAdd{487}\LinkedListPrint\LinkedListDraw

  \LinkedListSortedAdd{718}\LinkedListPrint\LinkedListDraw

  \LinkedListSortedAdd{962}\LinkedListPrint\LinkedListDraw

  \LinkedListDelete{487}\LinkedListPrint\LinkedListDraw

  \LinkedListDelete{732}\LinkedListPrint\LinkedListDraw

  \LinkedListDelete{500}

  \LinkedListDelete{962}\LinkedListPrint\LinkedListDraw

  \LinkedListDelete{718}\LinkedListPrint\LinkedListDraw
\end{Example}
\else
  \PSTricksExampleNotShown
\fi                             % End \pstrickstrue

\subsubsection{Associative arrays}

  To finish, we will give a complete solution to a classical problem: to count
the number of occurrences of the various letters in a sentence. For this, we
will first build some macros to deal with \emph{associative} arrays, as they
have been popularized by scripting languages like \texttt{AWK} and
\texttt{Perl}.

\begin{Example}
  \makeatletter

  % /HLComment?Internally, we use two "standard" arrays to define one associative array;
  \newarray\AssociativeArray@Names
  \newarray\AssociativeArray@Values

  \newcount\AssociativeArrayNbValues
  \AssociativeArrayNbValues=\z@

  \newif\ifAssociativeArray@ElementFound

  % /HLComment?To store one element;
  /HLMacro?\def\AssociativeArray;(#1)=#2{%
  \expandarrayelementtrue
  \AssociativeArray@ElementFoundfalse
  \edef\@tempa{#1}%
  \Multido{\iValue=\@ne+\@ne}{\AssociativeArrayNbValues}{%
    \checkAssociativeArray@Names(\iValue)%
    \ifx\@tempa\cachedata
      % /HLComment?This element already exist: we replace it current value;
      % /HLComment?\checkAssociativeArray@Values(\iValue)% Debug;
      % /HLComment?\typeout{In #1, replace `\cachedata'\space by `#2'}% Debug;
      \AssociativeArray@Values(\iValue)={#2}%
      \AssociativeArray@ElementFoundtrue
      \multidostop
    \fi}
  \ifAssociativeArray@ElementFound
  \else
    % /HLComment?New element;
    \advance\AssociativeArrayNbValues\@ne
    \AssociativeArray@Names(\AssociativeArrayNbValues)={#1}%
    \AssociativeArray@Values(\AssociativeArrayNbValues)={#2}%
  \fi}

  % /HLComment?To get one element;
  /HLMacro?\def\checkAssociativeArray;(#1){%
  \edef\@tempa{#1}%
  \edef\@tempb{999999}%
  \Multido{\iValue=\@ne+\@ne}{\AssociativeArrayNbValues}{%
    \checkAssociativeArray@Names(\iValue)%
    \ifx\@tempa\cachedata
      % /HLComment?We have found it by name;
      \edef\@tempb{\iValue}%
      \multidostop
    \fi}
  % /HLComment?We have now to get it value;
  \checkAssociativeArray@Values(\@tempb)}

  % /HLComment?Simple macro to print all the associative array;
  /HLMacro?\def\printAssociativeArray;{%
  \multido{\iValue=\@ne+\@ne}{\AssociativeArrayNbValues}{%
    \checkAssociativeArray@Names(\iValue)%
    \iValue: \BS\texttt{FirstNames(\cachedata)}=`\AssociativeArray@Values(\iValue)'\space}}

  \makeatother

  \let\FirstNames\AssociativeArray
  \let\checkFirstNames\checkAssociativeArray
  \let\printFirstNames\printAssociativeArray

  \FirstNames(Crawford)={Joan}
  \FirstNames(Tierney)={Gene}
  \FirstNames(Lake)={Veronika}

  \printFirstNames

  \checkFirstNames(Lake)
  \verb+\FirstNames(Lake)+=`\cachedata'

  \checkFirstNames(Monroe)
  \ifemptydata
    \verb+\FirstNames(Monroe) undefined!+
  \else
    \verb+\FirstNames(Monroe)+=`\cachedata'
  \fi
\end{Example}

\makeatletter

% Internally, we use two "standard" arrays to define one associative array
\newarray\AssociativeArray@Names
\newarray\AssociativeArray@Values

\newcount\AssociativeArrayNbValues
\AssociativeArrayNbValues=\z@

\newif\ifAssociativeArray@ElementFound

% To store one element
\def\AssociativeArray(#1)=#2{%
\expandarrayelementtrue
\AssociativeArray@ElementFoundfalse
\edef\@tempa{#1}%
\Multido{\iValue=\@ne+\@ne}{\AssociativeArrayNbValues}{%
  \checkAssociativeArray@Names(\iValue)%
  \ifx\@tempa\cachedata
    % This element already exist: we replace it current value
    % \checkAssociativeArray@Values(\iValue)% debug
    % \typeout{In #1, replace `\cachedata'\space by `#2'}% Debug
    \AssociativeArray@Values(\iValue)={#2}%
    \AssociativeArray@ElementFoundtrue
    \multidostop
  \fi}
\ifAssociativeArray@ElementFound
\else
  % New element
  \advance\AssociativeArrayNbValues\@ne
  \AssociativeArray@Names(\AssociativeArrayNbValues)={#1}%
  \AssociativeArray@Values(\AssociativeArrayNbValues)={#2}%
\fi}

% To get one element
\def\checkAssociativeArray(#1){%
\edef\@tempa{#1}%
\edef\@tempb{999999}%
\Multido{\iValue=\@ne+\@ne}{\AssociativeArrayNbValues}{%
  \checkAssociativeArray@Names(\iValue)%
  \ifx\@tempa\cachedata
    % We have found it by name
    \edef\@tempb{\iValue}%
    \multidostop
  \fi}
% We have now to get it value
\checkAssociativeArray@Values(\@tempb)}

% Simple macro to print all the associative array
\def\printAssociativeArray{%
\multido{\iValue=\@ne+\@ne}{\AssociativeArrayNbValues}{%
  \checkAssociativeArray@Names(\iValue)%
  \iValue: \BS\texttt{FirstNames(\cachedata)}=`\AssociativeArray@Values(\iValue)'}}

\makeatother

\vspace{2mm}
  Now, we can define the macros to read a sentence, to cut it letter by letter,
to count the occurrences of each of them, and finally to draw a summary plot of
the results.

\makeatletter                   % Problem with the \ifpstricks conditional if not set...

\ifpstricks                     % If PSTricks is available
\begin{CenterExample}
  \makeatletter

  % /HLComment?A \PerChar style macro adapted from Juergen Schlegelmilch;
  % /HLComment?(<schlegel@Informatik.Uni-Rostock.de> - posted on c.t.t. January 27, 1998);
  /HLMacro?\def\PerChar;#1#2{\DoPerChar#1#2\@nil}

  /HLMacro?\def\DoPerChar;#1#2#3\@nil{%
  #1#2%
  \edef\@tempa{#3}%
  \ifx\@tempa\empty
  \else
    \DoPerChar#1#3\@nil
  \fi}

  % /HLComment?The action to do for each character: we increase the counter;
  % /HLComment?for this character by one;
  /HLMacro?\def\ActionPerChar;#1{% /HLComment?#1=character;
  \checkAssociativeArray(#1)%
  \ifemptydata
    \@tempcnta=\@ne
  \else
    \@tempcnta=\cachedata
    \advance\@tempcnta\@ne
  \fi
  \AssociativeArray(#1)={\the\@tempcnta}}

  % /HLComment?To store in an associative array the number of occurrences by characters;
  % /HLComment?(spaces are not counted);
  /HLMacro?\def\StatSentence;#1{% /HLComment?#1=sentence;
  \AssociativeArrayNbValues=\z@
  \PerChar{\ActionPerChar}{#1}}

  % /HLComment?To draw a plot of the number of occurrences of the characters of a sentence;
  /HLMacro?\def\DrawOccurrences;#1{% /HLComment?#1=sentence;
  \StatSentence{#1}
  %
  \pst@cnta=\AssociativeArrayNbValues
  \advance\pst@cnta\@ne
  %
  % /HLComment?To know the maximum of the numbers;
  \pst@cntb=\z@
  \Multido{\iValue=\@ne+\@ne}{\AssociativeArrayNbValues}{%
    \checkAssociativeArray@Values(\iValue)%
    \ifnum\cachedata>\pst@cntb
      \pst@cntb=\cachedata
    \fi}
  \advance\pst@cntb\@ne
  %
  % /HLComment?The drawing itself;
  \pspicture(-0.5,-0.5)(\pst@cnta,\pst@cntb)
    \psaxes[axesstyle=frame,labels=y](\pst@cnta,\pst@cntb)
    \multido{\iValue=\@ne+\@ne}{\AssociativeArrayNbValues}{%
      \Draw@OneOccurrence{\iValue}}
  \endpspicture}

  % /HLComment?To draw the value for one occurrence;
  /HLMacro?\def\Draw@OneOccurrence;#1{% /HLComment?#1=Letter;
  \checkAssociativeArray@Values(#1)%
  \psdot(#1,\cachedata)
  \rput[B](\iValue,-0.5){\AssociativeArray@Names(#1)}}

  \makeatother

  \let\Names\AssociativeArray
  \let\checkNames\checkAssociativeArray

  \psset{dotstyle=o,dotsize=0.4}

  \DrawOccurrences{We are Wednesday}
\end{CenterExample}

\makeatletter

%% A \PerChar style macro adapted from Juergen Schlegelmilch
%% (<schlegel@Informatik.Uni-Rostock.de> - posted on c.t.t. Jan. 27, 1998)
\def\PerChar#1#2{\DoPerChar#1#2\@nil}

\def\DoPerChar#1#2#3\@nil{%
#1#2%
\edef\@tempa{#3}%
\ifx\@tempa\empty
\else
  \DoPerChar#1#3\@nil
\fi}

% The action to do for each character: we increase the counter
% for this character by one
\def\ActionPerChar#1{% #1=character
\checkAssociativeArray(#1)%
\ifemptydata
  \@tempcnta=\@ne
\else
  \@tempcnta=\cachedata
  \advance\@tempcnta\@ne
\fi
\AssociativeArray(#1)={\the\@tempcnta}}

% To store in an associative array the number of occurrences by characters
% (spaces are not counted)
\def\StatSentence#1{% #1=sentence
\AssociativeArrayNbValues=\z@
\PerChar{\ActionPerChar}{#1}}

% To draw a plot of the number of occurrences of the characters of a sentence
\def\DrawOccurrences#1{%
\StatSentence{#1}
%
\pst@cnta=\AssociativeArrayNbValues
\advance\pst@cnta\@ne
%
% To know the maximum of the numbers
\pst@cntb=\z@
\Multido{\iValue=\@ne+\@ne}{\AssociativeArrayNbValues}{%
  \checkAssociativeArray@Values(\iValue)%
  \ifnum\cachedata>\pst@cntb
    \pst@cntb=\cachedata
  \fi}
\advance\pst@cntb\@ne
%
% The drawing itself
\pspicture(-0.5,-0.5)(\pst@cnta,\pst@cntb)
  \psaxes[axesstyle=frame,labels=y](\pst@cnta,\pst@cntb)
  \multido{\iValue=\@ne+\@ne}{\AssociativeArrayNbValues}{%
    \Draw@OneOccurrence{\iValue}}
\endpspicture}

% To draw the value for one occurrence
\def\Draw@OneOccurrence#1{%
\checkAssociativeArray@Values(#1)%
\psdot(#1,\cachedata)
\rput[B](\iValue,-0.5){\AssociativeArray@Names(#1)}}

\let\Names\AssociativeArray
\let\checkNames\checkAssociativeArray

\psset{dotstyle=o,dotsize=0.4}

\makeatother

\vspace{8mm}                    % Poor hack to force a page break...
  We will now write another version, using another array to store indexes,
to allow us to sort the letters found in the sentence read.

\begin{CenterExample}
  \makeatletter

  % /HLComment?To get in #2 the ASCII code of the #1 character;
  /HLMacro?\def\Character@AsciiCode;#1#2{\chardef#2=`#1\relax}

  % /HLComment?A new "standard" array to store the sorted indexes;
  \newarray\AssociativeArray@Indexes

  % /HLComment?A simple macro to print the array of indexes (for debugging);
  /HLMacro?\def\PrintIndexes;{%
  \multido{\iIndexes=\@ne+\@ne}{\AssociativeArrayNbValues}{%
    \checkAssociativeArray@Indexes(\iIndexes)%
    \BS\texttt{indexes(\iIndexes)}=`\cachedata'\space}}

  % /HLComment?We redefine the macro to store an element in the associative array,;
  % /HLComment?to allow to sort it elements, using an additional array of indexes;
  /HLMacro?\def\AssociativeArray;(#1)=#2{%
  \expandarrayelementtrue
  \AssociativeArray@ElementFoundfalse
  \edef\@tempa{#1}%
  \edef\@tempb{#2}%
  \Multido{\iValue=\@ne+\@ne}{\AssociativeArrayNbValues}{%
    \checkAssociativeArray@Names(\iValue)%
    \ifx\@tempa\cachedata
      % /HLComment?This element already exist: we replace it current value;
      % /HLComment?\checkAssociativeArray@Values(\iValue)% Debug;
      % /HLComment?\typeout{In #1, replace `\cachedata'\space by `#2'}% Debug;
      \AssociativeArray@Values(\iValue)={\@tempb}%
      \AssociativeArray@ElementFoundtrue
      \multidostop
    \fi}
  \ifAssociativeArray@ElementFound
  \else
    % /HLComment?New element: we must insert it at it sorted position;
    \@tempcnta=\@ne
    \expandafter\Character@AsciiCode\@tempa\@tempx
    \Multido{\iValue=\@ne+\@ne}{\AssociativeArrayNbValues}{%
      \checkAssociativeArray@Names(\iValue)%
      \expandafter\Character@AsciiCode\cachedata\@tempy
      \ifnum\@tempx<\@tempy
        % /HLComment?The new element must be before this one: we will exchange their values;
        \checkAssociativeArray@Indexes(\iValue)%
        \@tempcntb=\cachedata
        \advance\@tempcntb\@ne
        \AssociativeArray@Indexes(\iValue)={\the\@tempcntb}%
      \else
        \advance\@tempcnta\@ne
      \fi}
    %
    \advance\AssociativeArrayNbValues\@ne
    \AssociativeArray@Names(\AssociativeArrayNbValues)={\@tempa}%
    \AssociativeArray@Values(\AssociativeArrayNbValues)={\@tempb}%
    \AssociativeArray@Indexes(\AssociativeArrayNbValues)={\the\@tempcnta}%
  \fi}

  % /HLComment?We change the way we draw each number of occurrences;
  /HLMacro?\def\Draw@OneOccurrence;#1{%
  \checkAssociativeArray@Indexes(#1)%
  \pst@cnta=\cachedata
  \checkAssociativeArray@Values(#1)%
  \psdot(\the\pst@cnta,\cachedata)
  \rput[B](\the\pst@cnta,-0.5){\AssociativeArray@Names(#1)}}

  \makeatother

  \DrawOccurrences{We are Wednesday}

  \PrintIndexes
\end{CenterExample}

\makeatletter

% To get in #2 the ASCII code of the #1 character
\def\Character@AsciiCode#1#2{\chardef#2=`#1\relax}%

% A new "standard" array to store the sorted indexes
\newarray\AssociativeArray@Indexes

% We redefine the macro to store an element in the associative array,
% to allow to sort it elements, using an additional array of indexes
\def\AssociativeArray(#1)=#2{%
\expandarrayelementtrue
\AssociativeArray@ElementFoundfalse
\edef\@tempa{#1}%
\edef\@tempb{#2}%
\Multido{\iValue=\@ne+\@ne}{\AssociativeArrayNbValues}{%
  \checkAssociativeArray@Names(\iValue)%
  \ifx\@tempa\cachedata
    % This element already exist: we replace it current value
    % \checkAssociativeArray@Values(\iValue)% Debug
    % \typeout{In #1, replace `\cachedata'\space by `#2'}% Debug
    \AssociativeArray@Values(\iValue)={\@tempb}%
    \AssociativeArray@ElementFoundtrue
    \multidostop
  \fi}
\ifAssociativeArray@ElementFound
\else
  % New element: we must insert it at it sorted position
  \@tempcnta=\@ne
  \expandafter\Character@AsciiCode\@tempa\@tempx
  \Multido{\iValue=\@ne+\@ne}{\AssociativeArrayNbValues}{%
    \checkAssociativeArray@Names(\iValue)%
    \expandafter\Character@AsciiCode\cachedata\@tempy
    \ifnum\@tempx<\@tempy
      % The new element must be before this one: we will exchange their values
      \checkAssociativeArray@Indexes(\iValue)%
      \@tempcntb=\cachedata
      \advance\@tempcntb\@ne
      \AssociativeArray@Indexes(\iValue)={\the\@tempcntb}%
    \else
      \advance\@tempcnta\@ne
    \fi}
  %
  \advance\AssociativeArrayNbValues\@ne
  \AssociativeArray@Names(\AssociativeArrayNbValues)={\@tempa}%
  \AssociativeArray@Values(\AssociativeArrayNbValues)={\@tempb}%
  \AssociativeArray@Indexes(\AssociativeArrayNbValues)={\the\@tempcnta}%
\fi}

% We change the way we draw each number of occurrences
\def\Draw@OneOccurrence#1{%
\checkAssociativeArray@Indexes(#1)%
\pst@cnta=\cachedata
\checkAssociativeArray@Values(#1)%
\psdot(\the\pst@cnta,\cachedata)
\rput[B](\the\pst@cnta,-0.5){\AssociativeArray@Names(#1)}}

\makeatother

\begin{CenterExample}
  \DrawOccurrences{Tuesday 8 February}
\end{CenterExample}

\noindent Now, we can easily redefine the internal macro which plot each
occurrence, to obtain various kinds of graphics:

\begin{CenterExample}
  \makeatletter

  /HLMacro?\def\Draw@OneOccurrence;#1{%
  \checkAssociativeArray@Indexes(#1)%
  \pst@cnta=\cachedata
  \checkAssociativeArray@Values(#1)
  \psline[linestyle=dotted](\the\pst@cnta,0)(\the\pst@cnta,\cachedata)
  \psdot(\the\pst@cnta,\cachedata)
  \rput[B](\the\pst@cnta,-0.5){\AssociativeArray@Names(#1)}}

  \makeatother

  \psset{xunit=0.5,dotstyle=*,dotsize=0.3}
  \DrawOccurrences{I do not believe that it can be correct}
\end{CenterExample}

\begin{CenterExample}
  \makeatletter

  \SpecialCoor

  /HLMacro?\def\Draw@OneOccurrence;#1{%
  \checkAssociativeArray@Indexes(#1)%
  \pst@cnta=\cachedata
  \checkAssociativeArray@Values(#1)%
  \pspolygon*[linecolor=red](! \the\pst@cnta\space 0.3 sub 0)
                            (! \the\pst@cnta\space 0.3 sub \cachedata)
                            (! \the\pst@cnta\space 0.3 add \cachedata)
                            (! \the\pst@cnta\space 0.3 add 0)
  \rput[B](\the\pst@cnta,-0.5){\AssociativeArray@Names(#1)}}

  \makeatother

  \psset{xunit=0.5,ticks=y}
  \DrawOccurrences{I don't know a better solution than this one}
\end{CenterExample}
\else
  \PSTricksExampleNotShown
\fi                             % End \pstrickstrue

\section{Thanks}

  I would like to thank Rolf Niepraschk \verb+<niepraschk@ptb.de>+ to have
carefully read a preliminary version of this document and to have sent me
several good advices to clarify and improve some points.

\begin{thebibliography}{9}
  \bibitem{SVB93} Stephan von  \textsc{Bechtolsheim}, \emph{\TeX{} in
  Practice}, four volumes, Springer-Verlag, New York, 1993 (see also
  \url{CTAN: macros/tip/arraymac.tip}).
  \bibitem{REPEAT} \RepeatPackage, by Victor \textsc{Eijkhout}, University of
  Illinois, Urbana-Champaign, USA, \url{CTAN: macros/generic/eijkhout/repeat}
  \bibitem{EG98} Philippe \textsc{Esperet} and Denis \textsc{Girou},
  \emph{Coloriage du pavage dit de Truchet}, Cahiers GUTenberg, Number~31,
  December 98, pages 5-18, in french (see
  \url{http://www.gutenberg.eu.org/pub/GUTenberg/publicationsPS/31-girou.ps.gz}).
  \bibitem{ALDRATEX} \AlDraTeXPackage, by Eitan M. \textsc{Gurari}, Ohio State
  University, Columbus, USA, \url{CTAN: graphics/dratex} (see also
  \url{http://www.cis.ohio-state.edu/~gurari/systems.html}).
  \bibitem{Gur94} Eitan M. \textsc{Gurari}, \emph{\TeX{} and \LaTeX: Drawing
  and Literate Programming}, McGraw-Hill, New-York, 1994.
  \bibitem{METAPOST} \MetaPostPackage, by John D. \textsc{Hobby}, AT\&T Bell
  Laboratories, USA, \url{CTAN: graphics/metapost} (see also
  \url{http://cm.bell-labs.com/who/hobby/MetaPost.html}).
  \bibitem{FORMLETT} \FormLettPackage, by Zhuhan \textsc{Jiang}, University of
  New England, Armidale, Australia, \url{CTAN: macros/generic/formlett.sty} (see
  also \url{http://maths.une.edu.au/~zhuhan/util.html#computer}).
  \bibitem{MULTIDO} \MultiDoPackage, by Timothy van \textsc{Zandt}, INSEAD,
  Fontainebleau, France, \url{CTAN: macros/generic/multido}
  \bibitem{PSTRICKS} \PSTricksPackage, by Timothy van \textsc{Zandt}, INSEAD,
  Fontainebleau, France, \url{CTAN: graphics/pstricks} (see also
  \url{http://www.tug.org/applications/PSTricks}).
\end{thebibliography}

\end{document}
